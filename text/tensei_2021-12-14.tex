\documentclass[12pt]{article}

\usepackage{CJKutf8}
\usepackage[T1]{fontenc}
\usepackage[overlap, CJK]{ruby}
\usepackage{setspace}

\renewcommand{\rubysep}{-1.5ex}
\doublespacing

\begin{document}
\begin{CJK}{UTF8}{ipxm}

    \ruby{衣料品}{いりょうひん}を扱う会社で\ruby{絶}{た}えず上司に\ruby{叱責}{しっせき}され、家に帰れば妻子にいたぶられる。
    昭和の\ruby{後期}{こうき}に人気を\ruby{博}{はく}したギャグ漫画『ダメおやじ』の主人公、雨野だめ助である。
    
    \ruby{連載}{れんさい}中、気の\ruby{毒}{どく}に思ったファンから\ruby{激励}{げきれい}の手紙が出版社に届いた。
    \ruby{当方}{とうほう}はまだ小学生だったが、仲間と毎週、\ruby{漫画誌}{まんがし}を回し読みしたことを覚えている。

    作者の\ruby{古谷}{ふるたに}\ruby{三敏}{みつとし}さんが85歳で亡くなった。
    \ruby{共著}{きょうちょ}『ボクの\ruby{満洲}{まんしゅう} 漫画家たちの\ruby{敗戦}{はいせん}体験』 などを読むと、子供時代を\ruby{奉天}{ほうてん}(今の\ruby{瀋陽}{しんよう})や\ruby{北京}{ぺきん}で過ごしている。
    \ruby{敗色}{はいしょく}が濃くなると、「\ruby{敵兵}{てきへい}が来たら、これで殺せ」と父親から\ruby{手榴弾}{しゅりゅうだん}の使い方を教えられる。
    中の良い同年代の中国人もいたが、大人の影響か、見下ろす感覚は隠せなかったという。

    引き揚げてからは「満洲、満洲」といじめられる。
    後ろめたさはぬぐえず、戦後も長く大陸中国を旅することができなかった。
    代わりに訪れた台湾で、街中の\ruby{崩}{くず}れたレンガを見た瞬間、北京での暮らしが\ruby{蘇}{よみがえ}り、\ruby{感極}{かんきわ}まったという。

    一方で、\ruby{万事}{ばんじ}おおらかだった大陸での生活は、漫画家としての仕事にも影響した。
    「漫画の締め切りも怖くない。
    親子三人の暮らしが\ruby{切羽詰}{せっぱつ}まっても\ruby{絶望}{ぜつぼう}しない。
    どこか平気なんです」。
    言われてみれば、ダメ助もどんな\ruby{窮地}{きゅうち}に\ruby{陥}{おちい}ろうと、どこか\ruby{泰然}{たいぜん}としていた。

    そんなダメ助の運命は後半に\ruby{突如}{とつじょ}、\ruby{上向}{うわむ}く。
    ついには\ruby{財閥}{ざいばつ}\ruby{令嬢}{れいじょう}に気に入られ社長に\ruby{就任}{しゅうにん}する。
    \ruby{波乱}{はらん}\ruby{万丈}{ばんじょう}の大陸暮らしで\ruby{培}{つちか}われた\ruby{楽観主義}{らっかんしゅぎ}が、\ruby{作風}{さくふう}でも生き方でも\ruby{大輪}{たいりん}の花を咲かせた。

\end{CJK}
\end{document}
