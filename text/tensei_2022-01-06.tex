\documentclass[12pt]{article}

\usepackage{CJKutf8}
\usepackage[T1]{fontenc}
\usepackage[overlap, CJK]{ruby}
\usepackage{setspace}

\renewcommand{\rubysep}{-1.5ex}
\doublespacing

\begin{document}
\begin{CJK}{UTF8}{ipxm}
    在日\ruby{米軍}{べいぐん}という存在が、日本政府や\ruby{自治体}{じちたい}の手の届かないところにある。
    そう痛感させた出来事の一つが、沖縄国際大学への米軍ヘリ\ruby{墜落}{ついらく}だった。
    2004年、大学本館に\ruby{激突}{げきとつ}して炎上し、\ruby{地元}{じもと}の消防がかけつけた。
    しかし火を消し止めた後、現場から締め出されてしまう。

    米軍が黄色いテープを張り巡らし、消防だけでなく警察にも現場検証を許さなかった。
    中に入れたのはピザの配達ぐらい。
    そんな理不尽さを許したのが日米地位協定である。
    米軍の特権を認めるこの協定は、オミクロン株の抜け穴にもなったようだ。

    日本政府はこの1月\ruby{余}{あま}り、外国人の新規\ruby{入国}{にゅうこく}を\ruby{停止}{ていし}するなど\ruby{水際}{みずぎわ}対策を強めてきた。
    しかし米軍関係者は地位協定により例外扱いで、米\ruby{本土}{ほんど}から直接、基地に入ることができる。
    基地の中では年末から集団感染が起きていた。
    
    それがオミクロン株かどうかの検査をさせてほしいという沖縄県の申し出も、個人情報\ruby{保護}{ほご}を理由に断られた。
    そうこうするうちに県内に感染が拡大し、政府は\ruby{蔓延}{まんえん}\ruby{防止}{ぼうし}\ruby{等}{とう}\ruby{重点}{じゅうてん}措置を使わざるを得なくなった。

    水際対策は時間を稼ぎ、医療体制などを\ruby{整}{ととの}えるための\ruby{手立}{てだ}てである。
    そこに政府のコントロールの及ばない部分があることは、明らかに検疫体制の欠陥だろう。
    しわ寄せは、基地のある地域を\ruby{直撃}{ちょくげき}する。

    \ruby{玉城}{たまぐすく}デニー\ruby{知事}{ちじ}が指摘する通り「\ruby{構造}{こうぞう}的な問題」だが、そこに切り込む姿勢は日米\ruby{双方}{そうほう}とも見られない。
    当局者たちの頭の中に、黄色いテーブが張られているのだろうか。

    

    
\end{CJK}
\end{document}
