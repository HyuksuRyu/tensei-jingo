\documentclass[12pt]{article}

\usepackage{CJKutf8}
\usepackage[T1]{fontenc}
\usepackage[overlap, CJK]{ruby}
\usepackage{setspace}

\renewcommand{\rubysep}{-1.5ex}
\doublespacing

\begin{document}
\begin{CJK}{UTF8}{ipxm}

    \section*{\centering\ruby{驚嘆}{きょうたん}の\ruby{万能人}{ばんのうじん} \space (2021.11.21)}
    「モナ・リザ」などの\ruby{絵画}{かいが}で知られるレオナルド・ダビンチは、\ruby{芸術}{げいじゅつ}だけでなく科学や技術の分野でも才能を発揮した。
    その\ruby{片鱗}{へんりん}は、30歳の時に職を求め、ミラノ\ruby{公国}{こうこく}の\ruby{君主}{くんしゅ}に書いた手紙からも伺える。

    自分には戦争に\ruby{役立つ}{やくだつ}技術があると訴え、「\ruby{運搬}{うんぱん}\ruby{容易}{ようい}な\ruby{大砲}{たいほう}」 「\ruby{堅牢}{けんろう}な戦車」などが作れるとした。
    平和なときには大\ruby{建築物}{けんちくぶつ}や\ruby{彫刻}{ちょうこく}を手掛けるし、絵の\ruby{技量}{ぎりょう}も「他の何びととでも御比較あれ」と記した。
    (『レオナルド・ダ・ヴィンチの\ruby{手記}{しゅき}』)

    興味の\ruby{赴く}{おもむく}まま、\ruby{天文学}{てんもんがく}や\ruby{解剖学}{かいぼうがく}などにも手を広げた人である。
    \ruby{後世}{こうせい}の人々から驚きを持って「\ruby{万能人}{ばんのうじん}」と呼ばれたのは、芸術も科学も専門化、\ruby{細分化}{さいぶんか}が進んだことの\ruby{裏返し}{うらがえし}だろう。
    もしかしたら野球の世界も、それに近いものがあるかもしれない。

    エンゼルスの大谷翔平選手がアメリカン・リーグの最優秀選手(MVP)に選ばれた。
    \ruby{満票}{まんぴょう}での選出という\ruby{快挙}{かいきょ}は、\ruby{万能}{ばんのう}ぶりへの\ruby{驚嘆}{きょうたん}からだろう。
    「\ruby{投打}{とうだ}兼任は成功しない」「\ruby{投手}{とうしゅ}が\ruby{盗塁}{とうるい}などもってのほか」といった常識を\ruby{覆し}{くつがえし}ていった。

    「本当に純粋にどこまでうまくなれるのかなと、頑張れたところが良かった」という大谷選手の弁は、ひたすら真っすぐである。
    \ruby{大}{だい}リーグではすでに、自分も\ruby{二刀流}{にとうりゅう}を、という選手が現れているという。
    おそらく未来の選手となる子供たちの間にも。

    何かのために別の何かを諦める。
    そんな生き方とは違う道を大谷選手が示してくれた。
    一人一人がもっと\ruby{欲張}{よくば}って、楽しんでもいいのだと。



\end{CJK}
\end{document}